\documentclass[a4paper,12pt]{extarticle}
\usepackage[utf8x]{inputenc}
\usepackage[T1,T2A]{fontenc}
\usepackage[russian]{babel}
\usepackage{hyperref}
\usepackage{indentfirst}
\usepackage{listings}
\usepackage{color}
\usepackage{here}
\usepackage{array}
\usepackage{multirow}
\usepackage{graphicx}

\usepackage{caption}
\renewcommand{\lstlistingname}{Программа} % заголовок листингов кода

\bibliographystyle{ugost2008ls}

\usepackage{listings}
\lstset{ %
extendedchars=\true,
keepspaces=true,
language=C,						% choose the language of the code
basicstyle=\footnotesize,		% the size of the fonts that are used for the code
numbers=left,					% where to put the line-numbers
numberstyle=\footnotesize,		% the size of the fonts that are used for the line-numbers
stepnumber=1,					% the step between two line-numbers. If it is 1 each line will be numbered
numbersep=5pt,					% how far the line-numbers are from the code
backgroundcolor=\color{white},	% choose the background color. You must add \usepackage{color}
showspaces=false				% show spaces adding particular underscores
showstringspaces=false,			% underline spaces within strings
showtabs=false,					% show tabs within strings adding particular underscores
frame=single,           		% adds a frame around the code
tabsize=2,						% sets default tabsize to 2 spaces
captionpos=t,					% sets the caption-position to top
breaklines=true,				% sets automatic line breaking
breakatwhitespace=false,		% sets if automatic breaks should only happen at whitespace
escapeinside={\%*}{*)},			% if you want to add a comment within your code
postbreak=\raisebox{0ex}[0ex][0ex]{\ensuremath{\color{red}\hookrightarrow\space}},
texcl=true,
inputpath=listings,                     % директория с листингами
}

\usepackage[left=2cm,right=2cm,
top=2cm,bottom=2cm,bindingoffset=0cm]{geometry}

%% Нумерация картинок по секциям
\usepackage{chngcntr}
\counterwithin{figure}{section}
\counterwithin{table}{section}

%%Точки нумерации заголовков
\usepackage{titlesec}
\titlelabel{\thetitle.\quad}
\usepackage[dotinlabels]{titletoc}

%% Оформления подписи рисунка
\addto\captionsrussian{\renewcommand{\figurename}{Рисунок}}
\captionsetup[figure]{labelsep = period}

%% Подпись таблицы
\DeclareCaptionFormat{hfillstart}{\hfill#1#2#3\par}
\captionsetup[table]{format=hfillstart,labelsep=newline,justification=centering,skip=-10pt,textfont=bf}

%% Путь к каталогу с рисунками
\graphicspath{{fig/}}



\begin{document}	% начало документа

% Титульная страница
\begin{titlepage}	% начало титульной страницы

	\begin{center}		% выравнивание по центру

		\large Санкт-Петербургский Политехнический Университет Петра Великого\\
		\large Институт компьютерных наук и технологий \\
		\large Кафедра компьютерных систем и программных технологий\\[6cm]
		% название института, затем отступ 6см
		
		\huge Телекоммуникационные технологии\\[0.5cm] % название работы, затем отступ 0,5см
		\large Отчет по лабораторной работе №2\\[0.1cm]
		\large Корреляция\\[5cm]

	\end{center}


	\begin{flushright} % выравнивание по правому краю
		\begin{minipage}{0.25\textwidth} % врезка в половину ширины текста
			\begin{flushleft} % выровнять её содержимое по левому краю

				\large\textbf{Работу выполнил:}\\
				\large Чугунов А.А.\\
				\large {Группа:} 33501/4\\
				
				\large \textbf{Преподаватель:}\\
				\large Богач Н.В.

			\end{flushleft}
		\end{minipage}
	\end{flushright}
	
	\vfill % заполнить всё доступное ниже пространство

	\begin{center}
	\large Санкт-Петербург\\
	\large \the\year % вывести дату
	\end{center} % закончить выравнивание по центру

\thispagestyle{empty} % не нумеровать страницу
\end{titlepage} % конец титульной страницы

\vfill % заполнить всё доступное ниже пространство


% Содержание
\include{ToC}


\section{Цель работы}
Познакомиться со средствами генерации и визуализации простых сигналов.


\section{Теоретическая информация}
Аналоговый сигнал с математической точки зрения представляет собой функцию (как правило - функцию времени), и при его дискретизации мы получаем отсчеты, являющиеся значениями этой функции, вычисленными в дискретные моменты времени. Поэтому для расчета дискретезированного сигнала необходимо прежде всего сформировать вектор дискретных значний времени. Сформировав его, можно вычислять значения сигнала, используя этот вектор в различных фомрулах.

\section{Ход выполнения работы}
Для начала научимся строить простейшие сигналы и изучим саму процедуру построения. Сформируем следующий сигнал при помощи языка \textit{Pyton}:


\lstinputlisting[
	label=code:Plot1,
	caption={Plot1.py},% для печати символ '_' требует выходной символ '\'
]{Plot1.py}
\parindent=1cm % командна \lstinputlisting сбивает параментры отступа
%Текст без отступа (следует за вставкой)%

Получаем следующие результаты(Рисунок.~\ref{pic:Plot1}):

\begin{figure}[H]
	\begin{center}
		\includegraphics[scale=0.7]{Plot1}
		\caption{Гармонический сигнал представленный различными графическими функциями} 
		\label{pic:Plot1} % название для ссылок внутри кода
	\end{center}
\end{figure}

У нас вышел  гаромонический сигнал, который затухает по экспоненте из-за домножения его на экспоненту.

Продолжая изучать графические возможности пакета, попробуем построить косинусы различной частоты(Рисунок.~\ref{pic:Plot2}):

\lstinputlisting[
	label=code:Plot2,
	caption={Plot2.py},% для печати символ '_' требует выходной символ '\'
]{Plot2.py}
\parindent=1cm % командна \lstinputlisting сбивает параментры отступа
%Текст без отступа (следует за вставкой)%

\begin{figure}[H]
	\begin{center}
		\includegraphics[scale=0.7]{Plot2}
		\caption{Косинусы различной частоты} 
		\label{pic:Plot2} % название для ссылок внутри кода
	\end{center}
\end{figure}

Довольно часто необходимо изучать сигналы, которые на разных интервалах времени задаются разными формулами, таким образом, есть необходимость в  рассмотрении построения кусочных зависимостей. Ниже представлены следущие импульсы(Рисунок.~\ref{pic:Plot3}):
\begin{itemize}
\item Экспоненциальный
\item Прямоугльный, центрированный относительно начала отсчета времени
\item Несимметричный треугольный импульс
\end{itemize}

Зададим их следующим кодом:

\lstinputlisting[
	label=code:Plot3,
	caption={Plot3.py},% для печати символ '_' требует выходной символ '\'
]{Plot3.py}
\parindent=1cm % командна \lstinputlisting сбивает параментры отступа
%Текст без отступа (следует за вставкой)%

\begin{figure}[H]
	\begin{center}
		\includegraphics[scale=0.7]{Plot3}
		\caption{Различные виды импульсов} 
		\label{pic:Plot3} % название для ссылок внутри кода
	\end{center}
\end{figure}

Рассмотрим различные одиночные импульсы. К сожалению, пакет signal языка Python не умеет строить одиночные импульсы, поэтому задавать такого вида импульсы будем вручную. Первый на очереди - прямоугольный импульс. С помощью сложения двух прямоугольных импульсов с разнополярными амплитудами получаем следующий Рисунок.~\ref{pic:Plot4}


\lstinputlisting[
	label=code:Plot4,
	caption={Plot4.py},% для печати символ '_' требует выходной символ '\'
]{Plot4.py}
\parindent=1cm % командна \lstinputlisting сбивает параментры отступа
%Текст без отступа (следует за вставкой)%

\begin{figure}[H]
	\begin{center}
		\includegraphics[scale=0.7]{Plot4}
		\caption{Пара разнополярных прямоугольных импульсов} 
		\label{pic:Plot4} % название для ссылок внутри кода
	\end{center}
\end{figure}

Аналогичную процедуру проводим и для треугольного импульса, так же задавая функцию вручную. Далее  используем нашу функцию для задания трапецевидного импульса(Рисунок.~\ref{pic:Plot5})

\lstinputlisting[
	label=code:Plot5,
	caption={Plot5.py},% для печати символ '_' требует выходной символ '\'
]{Plot5.py}
\parindent=1cm % командна \lstinputlisting сбивает параментры отступа
%Текст без отступа (следует за вставкой)%

\begin{figure}[H]
	\begin{center}
		\includegraphics[scale=0.7]{Plot5}
		\caption{Трапецевидный импульс} 
		\label{pic:Plot5} % название для ссылок внутри кода
	\end{center}
\end{figure}

Если же нам необходимо  сформировать сигнал, кототрый имеет ограниченный по частоте спектр, можем использовать следующую функцию: \begin{equation} y = sinc(x) = \frac{sin(\pi  x)}{\pi x} \end{equation}

Формируем радиоимпульс путем домножения прямоугольного импульса на косинус(Рисунок.~\ref{pic:Plot6_1}) и строим с помощью этой функции(1) амплитудный спектр(Рисунок.~\ref{pic:Plot6_2}).

\lstinputlisting[
	label=code:Plot6,
	caption={Plot6.py},% для печати символ '_' требует выходной символ '\'
]{Plot6.py}
\parindent=1cm % командна \lstinputlisting сбивает параментры отступа
%Текст без отступа (следует за вставкой)%

\begin{figure}[H]
	\begin{center}
		\includegraphics[scale=0.7]{Plot6_1}
		\caption{Одиночный радиоимпульс} 
		\label{pic:Plot6_1} % название для ссылок внутри кода
	\end{center}
\end{figure}

\begin{figure}[H]
	\begin{center}
		\includegraphics[scale=0.7]{Plot6_2}
		\caption{Амплитудный спектр} 
		\label{pic:Plot6_2} % название для ссылок внутри кода
	\end{center}
\end{figure}

Видим, что спектр оказался несимметричным относильно частоты заполнения радиоимпульса.

Продолжим изучения различных импульсов построив одиночный радиоимпульс с гауссовой огибающей при помощи встроенной функции из пакета signal(Рисунок.~\ref{pic:Plot7_1}). Так же построим его спектр((Рисунок.~\ref{pic:Plot7_2})).


\lstinputlisting[
	label=code:Plot7,
	caption={Plot7.py},% для печати символ '_' требует выходной символ '\'
]{Plot7.py}
\parindent=1cm % командна \lstinputlisting сбивает параментры отступа
%Текст без отступа (следует за вставкой)%

\begin{figure}[H]
	\begin{center}
		\includegraphics[scale=0.7]{Plot7_1}
		\caption{Гауссов радиоимпульс} 
		\label{pic:Plot7_1} % название для ссылок внутри кода
	\end{center}
\end{figure}

\begin{figure}[H]
	\begin{center}
		\includegraphics[scale=0.7]{Plot7_2}
		\caption{Амплитудный спектр} 
		\label{pic:Plot7_2} % название для ссылок внутри кода
	\end{center}
\end{figure}

Спектр был построен при помощи быстрого преобразования Фурье из пакета signal. Так же был подсчитан максимальный уровень спектра в децибелах на граничных частотах. Границы спектра отображены на рисунке двумя точками.

После изучения одиночных импульсов целесообразно изучить последовательности импульсов. Для примера рассмотрим последовательность треугольных импульсов с различными амплитудами и задержками:

\lstinputlisting[
	label=code:Plot12,
	caption={Plot12.py},% для печати символ '_' требует выходной символ '\'
]{Plot12.py}
\parindent=1cm % командна \lstinputlisting сбивает параментры отступа
%Текст без отступа (следует за вставкой)%

В отличие от MATLAB Python не имеет функции pulsetran. В связи с этим данную функцию опять же пришлось писать вручную. Результаты постороения отображены на Рисунок.~\ref{pic:Plot12}

\begin{figure}[H]
	\begin{center}
		\includegraphics[scale=0.7]{Plot12}
		\caption{Последовательность треугольных импульсов} 
		\label{pic:Plot12} % название для ссылок внутри кода
	\end{center}
\end{figure}

Рассмотрим последовательности, которые мы можем сгенерировать при помощи пакета signal:
\begin{itemize}
\item Прямоугольная(Рисунок.~\ref{pic:Plot8})
\item Пилооразная(Рисунок.~\ref{pic:Plot9})
\end{itemize}

\lstinputlisting[
	label=code:Plot8,
	caption={Plot8.py},% для печати символ '_' требует выходной символ '\'
]{Plot8.py}
\parindent=1cm % командна \lstinputlisting сбивает параментры отступа
%Текст без отступа (следует за вставкой)%
\begin{figure}[H]
	\begin{center}
		\includegraphics[scale=0.7]{Plot8}
		\caption{Последовательность прямоугольных импульсов} 
		\label{pic:Plot8} % название для ссылок внутри кода
	\end{center}
\end{figure}

\lstinputlisting[
	label=code:Plot9,
	caption={Plot9.py},% для печати символ '_' требует выходной символ '\'
]{Plot9.py}
\parindent=1cm % командна \lstinputlisting сбивает параментры отступа
%Текст без отступа (следует за вставкой)%

\begin{figure}[H]
	\begin{center}
		\includegraphics[scale=0.7]{Plot9}
		\caption{Последовательность пилообразных импульсов} 
		\label{pic:Plot9} % название для ссылок внутри кода
	\end{center}
\end{figure}

Функцию Дирихле так же пришлось писать вручную, так как ее нет в библиотеке.

\lstinputlisting[
	label=code:Plot10,
	caption={Plot10.py},% для печати символ '_' требует выходной символ '\'
]{Plot10.py}
\parindent=1cm % командна \lstinputlisting сбивает параментры отступа
%Текст без отступа (следует за вставкой)%

Так как функция Дирихле зависит от двух параметров - \textit{x} и \textit{n}, рассмотрим графики при \textit{n = 7  и n = 8}(Рисунок.~\ref{pic:Plot10_1} и Рисунок.~\ref{pic:Plot10_2}).

\begin{figure}[H]
	\begin{center}
		\includegraphics[scale=0.7]{Plot10_1}
		\caption{Функция Дирихле n = 7} 
		\label{pic:Plot10_1} % название для ссылок внутри кода
	\end{center}
\end{figure}

\begin{figure}[H]
	\begin{center}
		\includegraphics[scale=0.7]{Plot10_2}
		\caption{Функция Дирихле n = 8} 
		\label{pic:Plot10_2} % название для ссылок внутри кода
	\end{center}
\end{figure}

Далее рассмотрим сигналы с менющейся мгновенной частотой. Мгновенную частоту можем менять по различным законам, подавая на вход функции chirp следующие параметры:
\begin{itemize}
\item 'linear'  (Рисунок.~\ref{pic:Plot11_1})
\item 'quadratic' (Рисунок.~\ref{pic:Plot11_2})
\item 'logarithmic' (Рисунок.~\ref{pic:Plot11_3})
\end{itemize}

\lstinputlisting[
	label=code:Plot11,
	caption={Plot11.py},% для печати символ '_' требует выходной символ '\'
]{Plot11.py}
\parindent=1cm % командна \lstinputlisting сбивает параментры отступа
%Текст без отступа (следует за вставкой)%
 На графиках получаем спектрграммы - зависимость мгновенного амплитудного спектра сигнала от времени. Величина модуля спектральной функции отображается цветом в коориднатах "время - частота".

\begin{figure}[H]
	\begin{center}
		\includegraphics[scale=0.7]{Plot11_1}
		\caption{Спектрограмма сигнала при линейном законе изменения мгновенной частоты} 
		\label{pic:Plot11_1} % название для ссылок внутри кода
	\end{center}
\end{figure}

\begin{figure}[H]
	\begin{center}
		\includegraphics[scale=0.7]{Plot11_2}
		\caption{Спектрограмма сигнала при квадратичном законе изменения мгновенной частоты} 
		\label{pic:Plot11_2} % название для ссылок внутри кода
	\end{center}
\end{figure}

\begin{figure}[H]
	\begin{center}
		\includegraphics[scale=0.7]{Plot11_3}
		\caption{Спектрограмма сигнала при экспоненциальном законе изменения мгновенной частоты} 
		\label{pic:Plot11_3} % название для ссылок внутри кода
	\end{center}
\end{figure}


\section{Выводы}
Проделав лабораторную работу, рассмотрели различные типовые сигналы часто использующиеся в ЦОС. Научились пользоваться различными библиотеками языка Python и самим языком Python. Сделали первые шаги в изучении преобразований Фурье и постороении спектров сигналов. В лабораторной работе были затронуты дискретные сигналы, которые существенно отличаются от аналоговых. Были рассмотрены импульсы и последовательности импульсов.
\end{document}



