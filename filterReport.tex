\include{settings}
\usepackage{amsmath}

\begin{document}	% начало документа

% Титульная страница
\include{titlepage}

% Содержание
\include{ToC}


\section{Цель работы}
Изучить воздействие ФНЧ на тестовый сигнал с шумом.


\section{Теоретическая информация}
Линейный фильтр — динамическая система, применяющая некий линейный оператор ко входному сигналу для выделения или подавления определённых частот сигнала. Суть фильтрации заключается в пропускании сигнала через линейную цепочку с какими-то заданными параметрами для получения нужного нам конечного сигнала.
Пускай имеем два сигнала - начальный и конечный, которые имеею следующий вид:
\begin{equation}x(t) = A_x e^{j(2\pi ft+\phi_x)}\end{equation}
\begin{equation}y(t) = A_y e^{j(2\pi ft+\phi_y)}\end{equation}
Поделив конечный сигнал(2) на начальный(1), получаем частотную характеристику:
\begin{equation}G(f) = \frac{A_y}{A_x} e^{j(\phi_y - \phi_x)}\end{equation}
Спектр выходного сигнала можно посчитать по следующей формуле:
\begin{equation}Y(f) = X(f) G(f)\end{equation}
Таким образом, варьируя частотную характеристку линейной цепи можем получать на выходе различные сигналы.
Если же работать во временной области пренебрегая преобразованиями Фурье, можем пропустить через линейную цепль функцию Дирака. Спектр его равне единице.
Тогда получаем следующее выражение:
\begin{equation}Y(f) =  G(f)\end{equation}
Реакция цепи g(t) на функцию Дирака носит назавние импульсной характеристики. Импульсная характеристика связана с частотной следующими соотношениями:
\begin{equation}g(t) =  \int_{-\infty}^{\infty} G(f) e^{j(2\pi ft)}df \end{equation}
\begin{equation}G(t) =  \int_{0}^{\infty} g(t) e^{-j(2\pi ft)}dt \end{equation}
Таким образом, во временной области получаем следующую формулу:
\begin{equation}y(t) =  x(t) * g(t)\end{equation}

В данной работе будем пользоваться тремя фильтрами: медианным фильтром, фильтром скользящего среднего и фильтром Баттерворта.
Медианный фильтр и фильтр скользящего среднего являются простой интерполяцией. Работая с обоими фильтами неоходимо задаться некоторым окном значений. В случае фильтра скользяцего среднего, необходимо на каждом шаге из окна выбирать среднее арифметическое значений окна. В случае медианного фильтра, необходимо сортировать значения окна и выбирать его медиану, то есть значение, стоящее посередине.

Работая с фильтром Баттерворта, необходимо знать его частотную характеристку. Её можно получить из следующей формулы:
\begin{equation}G^2(\omega) =  \frac{G_0^2}{1 + (\frac{\omega}{\omega_c})^{2n}}\end{equation}
Здесь, n - порядок фильтра, Wc - частота среда, Go - коэффициент усиления по постоянной составляющей.
\section{Ход выполнения работы}
Для начала сгенерируем гармонический сигнал. В данной работе будет использовать синус.(Рисунок.~\ref{pic:signal})

\begin{figure}[H]
	\begin{center}
		\includegraphics[scale=0.7]{signal}
		\caption{Гармонический сигнал} 
		\label{pic:signal} % название для ссылок внутри кода
	\end{center}
\end{figure}

Теперь добавим шума в сигнал. Шумом будет служить синус, но более высокой частоты и более низкой амплитуды.(Рисунок.~\ref{pic:noisysignal})


\begin{figure}[H]
	\begin{center}
		\includegraphics[scale=0.7]{noisysignal}
		\caption{Зашумленный гармонический сигнал} 
		\label{pic:noisysignal} % название для ссылок внутри кода
	\end{center}
\end{figure}

Рассмотрим спектры обоих сигналов.(Рисунок.~\ref{pic:spectrumSig} и Рисунок.~\ref{pic:spectrumNoisSig})

\begin{figure}[H]
	\begin{center}
		\includegraphics[scale=0.7]{spectrumSig}
		\caption{Спектр гармонического сигнала} 
		\label{pic:spectrumSig} % название для ссылок внутри кода
	\end{center}
\end{figure}

\begin{figure}[H]
	\begin{center}
		\includegraphics[scale=0.7]{spectrumNoisSig}
		\caption{Спектр зашумленного гармонического сигнала} 
		\label{pic:spectrumNoisSig} % название для ссылок внутри кода
	\end{center}
\end{figure}

На спектре зашумелнного гармонического сигнала наблюдаем паразитую частоту, которая является частотой добавленных шумов. Теперь воспользуемся фильтром скользящего среднего. Здесь, окно - 50 дискретных значений.(Рисунок.~\ref{pic:averageFilter})

\begin{figure}[H]
	\begin{center}
		\includegraphics[scale=0.7]{averageFilter}
		\caption{Сигнал после фильтрации фильтром скользящего среднего} 
		\label{pic:averageFilter} % название для ссылок внутри кода
	\end{center}
\end{figure}

Получили неплохие результаты, однако на практике окно в 50 значений недопустимо, что говорит нам о том, что для нашей задачи данный фильтр не подходит. Рассмотрим спектр сигнала после фильтрации.(Рисунок.~\ref{pic:spectrumAver})

\begin{figure}[H]
	\begin{center}
		\includegraphics[scale=0.7]{spectrumAver}
		\caption{Спектр после фильтрации фильтром скользящего среднего} 
		\label{pic:spectrumAver} % название для ссылок внутри кода
	\end{center}
\end{figure}

Однако в частотной области паразитная частота практически исчезла.
Теперь воспользуемся медианным фильтром. Здесь окно - 85 значений, что тоже недопустимо на практике. Результат фильтрации(Рисунок.~\ref{pic:medFilter}) и спектр(Рисунок.~\ref{pic:spectrumMed}).

\begin{figure}[H]
	\begin{center}
		\includegraphics[scale=0.7]{medFilter}
		\caption{Сигнал после фильтрации медианным фильтром} 
		\label{pic:medFilter} % название для ссылок внутри кода
	\end{center}
\end{figure}

\begin{figure}[H]
	\begin{center}
		\includegraphics[scale=0.7]{spectrumMed}
		\caption{Спектр после фильтрации медианным фильтром} 
		\label{pic:spectrumMed} % название для ссылок внутри кода
	\end{center}
\end{figure}

Опять получили не самые плохие результаты, более того, в частотной области паразитная частота была удалена, однако на практите медианный фильтр применять не будем для решения такого рода задач.

Переходим к последнему фильтру - фильтру Баттерворта. Задавшись частотой и порядком фильтра получили следующий сигнал(Рисунок.~\ref{pic:butterFilter}) и спектр(Рисунок.~\ref{pic:spectrumButter}).

\begin{figure}[H]
	\begin{center}
		\includegraphics[scale=0.7]{butterFilter}
		\caption{Сигнал после фильтрации фильтром Баттерворта} 
		\label{pic:butterFilter} % название для ссылок внутри кода
	\end{center}
\end{figure}

\begin{figure}[H]
	\begin{center}
		\includegraphics[scale=0.7]{spectrumButter}
		\caption{Спектр после фильтрации  фильтром Баттерворта} 
		\label{pic:spectrumButter} % название для ссылок внутри кода
	\end{center}
\end{figure}

Получили отличный результаты и в частотной и во временной области. Фильтр справился со своей задачей и отфильтровал шумы из нашего сигнала.
Код программы представлен ниже. Фильтр скользящего среднего пришлось реализовывать самостоятельно в виду его отсутствия в библиотеках языка Python.

\lstinputlisting[
	label=code:CorrelLab,
	caption={LinearFiltration.py},% для печати символ '_' требует выходной символ '\'
]{LinearFiltration.py}
\parindent=1cm % командна \lstinputlisting сбивает параментры отступа
%Текст без отступа (следует за вставкой)%

\section{Выводы}
Проделав лабораторную работу, рассмотрели понятие линейной фильтрации и научились пользоваться тремя фильтрами. Установили, что фильтр Баттерворта справляется с задачей фильтрации шумов лучше, чем это делают медианный фильтр и фильтр скользящего среднего. Выяснили, что отфильтровать сигнал можно при помощи грамотного подбора параметров для частотной характеристики линейной цепи. Подобрав верные параметры, получаем возможность полностью убрать зашумляющие сигналы, частоты которых выше заданной нами частосты среза. Неполное удаление шума фильтром возможно из-за пологости частотной характеристики на полосе подавления, однако, меняя порядок фильтра, этой проблемы можно избежать.
\end{document}



