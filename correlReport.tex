\include{settings}
\usepackage{amsmath}

\begin{document}	% начало документа

% Титульная страница
\include{titlepage}

% Содержание
\include{ToC}


\section{Цель работы}
Познакомиться с понятием коррелиции и функцией корреляции. 


\section{Теоретическая информация}
В данной лабораторной работе будем рассматривать корреляционный анализ. Его смысл состоит в количественном измерении степени сходства различных сигналов. Для этого будем использовать специальные корреляционные функции. Так, для получения взаимной корреляции двух последовательностей, имеем следующую формулу:
\begin{equation} r_{12}(j) = \frac{1}{N} \sum\limits_{n = 0}^{N - 1} x_1(n) x_2(n + j) = r_{12}(-j) =  \frac{1}{N} \sum\limits_{n = 0}^{N - 1} x_2(n) x_1(n - j) \end{equation}
Здесь j - это смещение одного сигнала относительно другого.

Так же можно ввести аналогичную формулу для непрерывной временной области:
\begin{equation} r_{12}(\tau) =  \lim_{T \to \infty} \frac{1}{T} \int_{-T/2}^{T/2} x_1(t)x_2(t + \tau) dt  \end{equation}

Рассчет корреляции можно ускорить, воспользовавшись следующей формулой:
\begin{equation} r_{12}(j) = \frac{1}{N}F_D^{-1}[X_1^*(k)X_2(k)] \end{equation}
Здесь, $F_D^{-1}$ - обратное преобразование Фурье.
При различной длине сигналов выполнить расчет корреляции можно путем добавления к двум последовательностям дополняющих нулей. Если последовательность $x_1(n)$ имеет
длину N1, а последовательность $x_2(n)$ — N2, то$x_1(n)$ дополняется (N2 − 1) нулями,
а $x_2(n)$ — (N1 − 1) нулями. Далее на основе двух расширенных последовательностей
рассчитывается взаимная корреляция.

\section{Ход выполнения работы}
Имеем сигнал, сосотоящий из нолей и единиц - [0, 0, 0, 1, 0, 1, 0, 1, 1, 1, 0, 0, 0, 0, 1, 0] и синхропосылку - [1, 0, 1]. Задача - найти положение синхропосылки в сигнале. Изначально преобразуем все ноли в -1. Это необходимо для корректной работы алгоритма быстрого расчета корреляции сигналов, так как сигнал будет дополняться нолями. Корреляцию рассчитаем с помощью встроенной функции correlate библиотеки numpy. Алгоритм для быстрого расчета корреляции напишем самостоятельно используя преобразования Фурье из той же библотеки. Результаты представлены на Рисунке.~\ref{pic:result}

\lstinputlisting[
	label=code:CorrelLab,
	caption={CorrelLab.py},% для печати символ '_' требует выходной символ '\'
]{CorrelLab.py}
\parindent=1cm % командна \lstinputlisting сбивает параментры отступа
%Текст без отступа (следует за вставкой)%

\begin{figure}[H]
	\begin{center}
		\includegraphics[scale=0.7]{result}
		\caption{Результаты работы прямого и быстрого расчета корреляции} 
		\label{pic:result} % название для ссылок внутри кода
	\end{center}
\end{figure}

Наблюдаем более быструю работу прямого алогоритма, что является очень странным. Можем сделать предположение, что быстрый алгоритм работает эффективней с большим количеством данных.
Стоит отметить, что оба алгоритма выполнили свою задачу и определили положение синхропосылки. В данном случае начало синхропосылки отмечено цифрой 3.


\section{Выводы}
Проделав лабораторную работу, рассмотрели понятие корреляции и научились пользоваться двумя алгоритмама её расчета. Установили, что прямой алогритм поиска корреляции более быстро рассчитывает корреляцию, чем быстрый алгоритм.
\end{document}



