\documentclass[a4paper,12pt]{extarticle}
\usepackage[utf8x]{inputenc}
\usepackage[T1,T2A]{fontenc}
\usepackage[russian]{babel}
\usepackage{hyperref}
\usepackage{indentfirst}
\usepackage{listings}
\usepackage{color}
\usepackage{here}
\usepackage{array}
\usepackage{multirow}
\usepackage{graphicx}

\usepackage{caption}
\renewcommand{\lstlistingname}{Программа} % заголовок листингов кода

\bibliographystyle{ugost2008ls}

\usepackage{listings}
\lstset{ %
extendedchars=\true,
keepspaces=true,
language=C,						% choose the language of the code
basicstyle=\footnotesize,		% the size of the fonts that are used for the code
numbers=left,					% where to put the line-numbers
numberstyle=\footnotesize,		% the size of the fonts that are used for the line-numbers
stepnumber=1,					% the step between two line-numbers. If it is 1 each line will be numbered
numbersep=5pt,					% how far the line-numbers are from the code
backgroundcolor=\color{white},	% choose the background color. You must add \usepackage{color}
showspaces=false				% show spaces adding particular underscores
showstringspaces=false,			% underline spaces within strings
showtabs=false,					% show tabs within strings adding particular underscores
frame=single,           		% adds a frame around the code
tabsize=2,						% sets default tabsize to 2 spaces
captionpos=t,					% sets the caption-position to top
breaklines=true,				% sets automatic line breaking
breakatwhitespace=false,		% sets if automatic breaks should only happen at whitespace
escapeinside={\%*}{*)},			% if you want to add a comment within your code
postbreak=\raisebox{0ex}[0ex][0ex]{\ensuremath{\color{red}\hookrightarrow\space}},
texcl=true,
inputpath=listings,                     % директория с листингами
}

\usepackage[left=2cm,right=2cm,
top=2cm,bottom=2cm,bindingoffset=0cm]{geometry}

%% Нумерация картинок по секциям
\usepackage{chngcntr}
\counterwithin{figure}{section}
\counterwithin{table}{section}

%%Точки нумерации заголовков
\usepackage{titlesec}
\titlelabel{\thetitle.\quad}
\usepackage[dotinlabels]{titletoc}

%% Оформления подписи рисунка
\addto\captionsrussian{\renewcommand{\figurename}{Рисунок}}
\captionsetup[figure]{labelsep = period}

%% Подпись таблицы
\DeclareCaptionFormat{hfillstart}{\hfill#1#2#3\par}
\captionsetup[table]{format=hfillstart,labelsep=newline,justification=centering,skip=-10pt,textfont=bf}

%% Путь к каталогу с рисунками
\graphicspath{{fig/}}

\usepackage{amsmath}

\begin{document}	% начало документа

% Титульная страница
\begin{titlepage}	% начало титульной страницы

	\begin{center}		% выравнивание по центру

		\large Санкт-Петербургский Политехнический Университет Петра Великого\\
		\large Институт компьютерных наук и технологий \\
		\large Кафедра компьютерных систем и программных технологий\\[6cm]
		% название института, затем отступ 6см
		
		\huge Телекоммуникационные технологии\\[0.5cm] % название работы, затем отступ 0,5см
		\large Отчет по лабораторной работе №2\\[0.1cm]
		\large Корреляция\\[5cm]

	\end{center}


	\begin{flushright} % выравнивание по правому краю
		\begin{minipage}{0.25\textwidth} % врезка в половину ширины текста
			\begin{flushleft} % выровнять её содержимое по левому краю

				\large\textbf{Работу выполнил:}\\
				\large Чугунов А.А.\\
				\large {Группа:} 33501/4\\
				
				\large \textbf{Преподаватель:}\\
				\large Богач Н.В.

			\end{flushleft}
		\end{minipage}
	\end{flushright}
	
	\vfill % заполнить всё доступное ниже пространство

	\begin{center}
	\large Санкт-Петербург\\
	\large \the\year % вывести дату
	\end{center} % закончить выравнивание по центру

\thispagestyle{empty} % не нумеровать страницу
\end{titlepage} % конец титульной страницы

\vfill % заполнить всё доступное ниже пространство


% Содержание
\include{ToC}


\section{Цель работы}
Познакомиться с понятием коррелиции и функцией корреляции. 


\section{Теоретическая информация}
В данной лабораторной работе будем рассматривать корреляционный анализ. Его смысл состоит в количественном измерении степени сходства различных сигналов. Для этого будем использовать специальные корреляционные функции. Так, для получения взаимной корреляции двух последовательностей, имеем следующую формулу:
\begin{equation} r_{12}(j) = \frac{1}{N} \sum\limits_{n = 0}^{N - 1} x_1(n) x_2(n + j) = r_{12}(-j) =  \frac{1}{N} \sum\limits_{n = 0}^{N - 1} x_2(n) x_1(n - j) \end{equation}
Здесь j - это смещение одного сигнала относительно другого.

Так же можно ввести аналогичную формулу для непрерывной временной области:
\begin{equation} r_{12}(\tau) =  \lim_{T \to \infty} \frac{1}{T} \int_{-T/2}^{T/2} x_1(t)x_2(t + \tau) dt  \end{equation}

Рассчет корреляции можно ускорить, воспользовавшись следующей формулой:
\begin{equation} r_{12}(j) = \frac{1}{N}F_D^{-1}[X_1^*(k)X_2(k)] \end{equation}
Здесь, $F_D^{-1}$ - обратное преобразование Фурье.
При различной длине сигналов выполнить расчет корреляции можно путем добавления к двум последовательностям дополняющих нулей. Если последовательность $x_1(n)$ имеет
длину N1, а последовательность $x_2(n)$ — N2, то$x_1(n)$ дополняется (N2 − 1) нулями,
а $x_2(n)$ — (N1 − 1) нулями. Далее на основе двух расширенных последовательностей
рассчитывается взаимная корреляция.

\section{Ход выполнения работы}
Имеем сигнал, сосотоящий из нолей и единиц - [0, 0, 0, 1, 0, 1, 0, 1, 1, 1, 0, 0, 0, 0, 1, 0] и синхропосылку - [1, 0, 1]. Задача - найти положение синхропосылки в сигнале. Изначально преобразуем все ноли в -1. Это необходимо для корректной работы алгоритма быстрого расчета корреляции сигналов, так как сигнал будет дополняться нолями. Корреляцию рассчитаем с помощью встроенной функции correlate библиотеки numpy. Алгоритм для быстрого расчета корреляции напишем самостоятельно используя преобразования Фурье из той же библотеки. Результаты представлены на Рисунке.~\ref{pic:result}

\lstinputlisting[
	label=code:CorrelLab,
	caption={CorrelLab.py},% для печати символ '_' требует выходной символ '\'
]{CorrelLab.py}
\parindent=1cm % командна \lstinputlisting сбивает параментры отступа
%Текст без отступа (следует за вставкой)%

\begin{figure}[H]
	\begin{center}
		\includegraphics[scale=0.7]{result}
		\caption{Результаты работы прямого и быстрого расчета корреляции} 
		\label{pic:result} % название для ссылок внутри кода
	\end{center}
\end{figure}

Наблюдаем более быструю работу прямого алогоритма, что является очень странным. Можем сделать предположение, что быстрый алгоритм работает эффективней с большим количеством данных.
Стоит отметить, что оба алгоритма выполнили свою задачу и определили положение синхропосылки. В данном случае начало синхропосылки отмечено цифрой 3.


\section{Выводы}
Проделав лабораторную работу, рассмотрели понятие корреляции и научились пользоваться двумя алгоритмама её расчета. Установили, что прямой алогритм поиска корреляции более быстро рассчитывает корреляцию, чем быстрый алгоритм.
\end{document}



